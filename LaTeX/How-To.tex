\documentclass{article}
%\usepackage{enumitem,amssymb}
%\newlist{todolist}{itemize}{2}
%\setlist[todolist]{label=$\square$
\usepackage{verbatim}
\usepackage{etoolbox}
\usepackage{datatool}


\title{SimEP How-To Guide}
\author{Svea \& Vincent}
\date{April 2025}

\newcommand{\KOM}{Svea}
\newcommand{\EP}{Vincent}
\newcommand{\city}{München}

\newcommand{\checkCity}{%
    \ifdefined\city%
        \ifthenelse{\equal{\city}{München}}{%
            \newcommand{\STADTVERTRETER}{Florian Kraus (Stadtschulrat)}%
            \newcommand{\EUVERTRETER}{Maria Noichl (MdEP)}%
            \newcommand{\THEMA}{Asyl- und Migrationspolitik}%
            \newcommand{\VERANSTALTUNGSORT}{Landtag Bayern}%
            \newcommand{\FOERDERER}{Europe Direct München, Stadt München \& Landtag Bayern}%
            \newcommand{\ausschussOne}{BUDG}%
            \newcommand{\ausschussTwo}{DROI}%
            \newcommand{\ausschussThree}{EMPL}%
            \newcommand{\ausschussFour}{LIBE}%
            \newcommand{\timeVorb}{07:45-09:00}%
            \newcommand{\timeEinf}{09:00-09:45}%
            \newcommand{\timeFrakOne}{09:45-11:15}%
            \newcommand{\timePauseOne}{11:15-11:30}%
            \newcommand{\timeAuss}{11:30-12:45}%
            \newcommand{\timeMittag}{12:45-13:15}%
            \newcommand{\timeFrakTwo}{13:15-13:45}%
            \newcommand{\timePauseTwo}{13:45-14:00}%
            \newcommand{\timePlenar}{14:00-15:00}%
        }{%
            \newcommand{\STADTVERTRETER}{Dr. Andrea Heilmaier (Wirtschafts- und Wissenschaftsreferentin)}%
            \newcommand{\EUVERTRETER}{Monika Hohlmeier (MdEP)}%
            \newcommand{\THEMA}{Außen- und Sicherheitspolitik}%
            \newcommand{\VERANSTALTUNGSORT}{Rathaus Nürnberg}%
            \newcommand{\FOERDERER}{Europe Direct Nürnberg \& Stadt Nürnberg}%
            \newcommand{\ausschussOne}{BUDG}%
            \newcommand{\ausschussTwo}{LIBE}%
            \newcommand{\ausschussThree}{SEDE}%
            \newcommand{\timeVorb}{07:45-09:00}%
            \newcommand{\timeEinf}{09:00-09:45}%
            \newcommand{\timeFrakOne}{09:45-11:15}%
            \newcommand{\timePauseOne}{11:15-11:30}%
            \newcommand{\timeAuss}{11:30-12:45}%
            \newcommand{\timeMittag}{12:45-13:15}%
            \newcommand{\timeFrakTwo}{13:15-13:45}%
            \newcommand{\timePauseTwo}{13:45-14:00}%
            \newcommand{\timePlenar}{14:00-15:00}%
        }%
    \fi%
}


\begin{document}
\checkCity

\begin{comment}
	\maketitle
	\newpage
	
	\section{Planung}
	\subsection{Finanzierung}
    \begin{itemize}
        \item Nächstes Jahr:
        \begin{itemize}
            \item Bamberg \& Bayreuth über Landeszentrale
            \item München \& Nürnberg über Staatskanzlei
        \end{itemize}
        \item Weitere Möglichkeiten:
        \begin{itemize}
            \item djo
            \item Staatskanzlei
            \item EU/EP
            \item Sponsoring
            \item Stadt/Europe Direct
        \end{itemize}
    \end{itemize}

	\subsection{Veranstaltungsort \& -tag}
    \begin{itemize}
        \item Ferien \& Prüfungen beachten
    \end{itemize}

    \subsection{Gastredner:innen}
    \begin{itemize}
        \item Direkt nachdem der Termin feststeht
        \item EU- \& Stadt-Vertreter
    \end{itemize}
    
	\subsection{Schulklassen}
     \begin{itemize}
        \item Telefonnummern der Lehrer anfragen!
        \item Reminder ca. einen Monat vorher
    \end{itemize}
    \begin{table}[h]
        \centering
        \begin{tabular}{c|c|c|c|c}
             Schule & Klasse & Lehrer & Anzahl Schüler & Namensliste erhalten\\
             \hline
             HSG Nürnberg & 11. & Fr. Maier-Hofer & 29 & Nein\\
             CEG Erlangen & ??? & Hr. Peters & 18 & Nein \\
             PPG Hersbruck & 11. & Fr. Raub & 20 & Nein \\
             Wirtschaftsschule Erlangen & 10. & Fr. Foldyna & 17 & Nein \\
        \end{tabular}
    \end{table}
	
	\subsection{Teamer:innen \& Unterkunft}
    \begin{itemize}
        \item Extra WhatsApp-Gruppe gründen (nur wirklich Teilnehmende)
        \item Hotel anfragen
        \item 2-3 Monate vorher fixe Zusage einholen
        \item dann buchen (je nach Stornierungsbedingungen)
        \item vor Frist nochmal nachfragen
    \end{itemize}
    \begin{table}[h]
        \centering
        \begin{tabular}{c|c|c|c}
             Fraktion & Leitung & weitere Teamer & Anzahl Schüler \\
             \hline
             EVP & & & \\
             S\&D & & & \\
             Renew & & & \\
             Grüne & & & \\
             ID & & &
        \end{tabular}

    \end{table}
    
    \begin{table}[h]
        \centering
        \begin{tabular}{c|c|c}
             Ausschuss & Leitung & weitere Teamer \\
             \hline
             Ausschuss I & & \\
             Ausschuss II & & \\
             Ausschuss III & & \\
             Ausschuss IV & & 
        \end{tabular}
    \end{table}
 
	\subsection{Unterlagen}
	\begin{itemize}
		\item Mappen:
		\begin{itemize}
			\item Fotobändchen
			\item Flyer
			\item Fraktionspapiere
			\item Gesetzesentwürfe
			\item Länderpapiere
			\item Teilnahmebescheinigungen
		\end{itemize}
		\item Namensschilder (Team \& Schüler)
		\item Raumschilder		
	\end{itemize}
 
	\subsection{Catering}
    \begin{itemize}
        \item über Stadt
    \end{itemize}

    \subsection{Abendessen}
    
	\subsection{Presse}
	\begin{itemize}
		\item Pressemitteilung
		\item Einladung $\to$ direkt Journalisten anschreiben?
	\end{itemize}   
	\newpage
\end{comment}
 
	\section{Ablauf}
	\subsection{Vorbereitung (\timeVorb)}
	\begin{itemize}
		\item Raumtour \& Raumschilder aushängen
        \item Technikcheck in Fraktionsräumen [Fraktionsteamer:innen]
		\item Technikcheck Plenum [\KOM\ \& \EP]
		\begin{itemize}
			\item fester Laptop
			\item Standby-Modus aus
			\item alle (und nur) nötigen Programme offen \& laufen
		\end{itemize}
		\item Dekoration Plenum \& Mappen abzählen [Fraktionsteamer:innen]
		\item Schüler in Empfang nehmen [\KOM\ \& \EP]
		%\item EU-VERTRETER in Empfang nehmen [\KOM\ \& \EP]		
	\end{itemize}
	
	\subsection{Einführung (\timeEinf)}
	\begin{itemize}
		\item Kurze Begrüßung/Intro [\KOM]
		\item Grußwort [\EUVERTRETER]
		\item Präsentation Briefing [\KOM]
        \item SuS-Anzahl nachzählen \& Mappen entsprechend anpassen
        \item Aufgaben:
        \begin{itemize}
            \item Fraktionsleiter am Podium
            \item Rest verteilt Mappen auf Stichwort
        \end{itemize}
    \end{itemize}
	
	\subsection{1. Fraktionssitzung (\timeFrakOne)}
	\begin{enumerate}
		\item Icebreaker (Bildkarten) [5 min]: \newline Assoziationen mit Fraktion \& Thema
        \item Zufällig in drei gleichgroße Kleingruppen einteilen
        \item Jeder Kleingruppe durch Namensschilder Ausschuss zuweisen (einen der Namensschilderbögen geben)
		\item Präsentation [15 min]
		\begin{itemize}
			\item Rolle der Schüler deutlich machen: \newline
            Nicht persönliche Meinung vertreten, aber auch kein Fraktionszwang
			\item Überblick über Fraktion
			\item Weltanschauung der Fraktion
			\item Rolle der Nationalität/Länderpapiere
			\item Weiteren Ablauf des Gesetzgebungsprozesses/der SimEP klarstellen
            \item Raum für Fragen
		\end{itemize}
		\item Diskussion über Position der Fraktion zu \THEMA \newline [10 min]
		\item In den Auschussgruppen [50 min]
		\begin{enumerate}
			\item Gesetzesentwurf (besonders zugewiesener Abschnitt) lesen
			\item EINEN (!) ÄA pro Ausschussgruppe erarbeiten \newline (muss sich auf einen inhaltlichen Aspekt beschränken)
	        \item Ausschusssprecher:in festlegen		
            \item ÄA in Antragsgrün einbringen
		\end{enumerate} 
		\item Wahl Fraktionsvorsitz [15 min]
		\begin{itemize}
			\item Vorsitz: Abschlussrede (ca. 1 Min.)
			\item Vize: Vorstellung (ca. 30 Sek.) und Reaktion (ca. 10 Sek.) \newline bei finalen ÄA 
			\item Namen in Worddokument (Fraktionen/Fraktionsvorsitzende) \newline eintragen
		\end{itemize}
	\end{enumerate}
	
	\subsection{Zwischenpause (\timePauseOne)}
	\begin{itemize}
		\item Reihenfolge ÄA festlegen (weitreichendste zuerst) [Ausschussteamer:innen]
	\end{itemize}
	
	\subsection{Ausschusssitzung (\timeAuss)}
	\begin{enumerate}
		\item Präsentation: Ausschuss und Ablauf vorstellen \& erklären [5 min]
        \item Vorstellen der ÄA [10 min] \newline
        für jeden ÄA:
        \begin{enumerate}
            \item kurze Präsentation ÄA durch Ausschusssprecher:in
			\item kurze Bedenkzeit, dann Reaktion der anderen Fraktionen (startend links von vorstellender Fraktion $\to$ Uhrzeigersinn)
            \item mglw. Erwiderung einbringende Fraktion/weitere Diskussion \newline (Redezeitbegrenzung [bspw. 1 Min.] sinnvoll)
        \end{enumerate}
		\item Informelle Verhandlungen [40 min]
		\begin{itemize}
			\item Mehrheitsfindung/Koalitionen schmieden
			\item Kompromissfindung $\to$ ÄA können noch abgeändert/zusammengelegt werden, um mehrheitsfähig zu sein
		\end{itemize}
		\item Abstimmung über ÄA [20 min]\newline
		Für jeden ÄA:
		\begin{itemize}
			\item Abstimmung
			\item Zählen und Eintragen in Antragsgrün \newline
            (Angenommen, wenn mehr JA als NEIN Stimmen \newline [Enthaltungen zählen nicht])
            \item Angenommene ÄA in Gesetzesentwurf übernehmen (Ausschuss mit eintragen)
		\end{itemize}
	\end{enumerate}
	
	\subsection{Mittagspause (\timeMittag)}
	\begin{itemize}
		\item Schüler:innen zu Plenarsaal bringen
	\end{itemize}
	
	\subsection{2. Fraktionssitzung (\timeFrakTwo)}
	\begin{enumerate}
		\item Bericht der Schüler:innen aus ihren Ausschüssen [15 min]
		\item Finalen ÄA überlegen \& in Antragsgrün einbringen [15 min] \newline Bspw.:
		\begin{itemize}
			\item Zurückgewiesener ÄA in Ausschuss (mglw. abgeschwächt)
			\item Rücknahme eines im Ausschuss beschlossenen ÄA
			\item Überschüssige Idee aus 1. Sitzung \newline (eine Ausschussgruppe hatte mehr als eine ÄA-Idee)
		\end{itemize}
	\end{enumerate}
	
	\subsection{Zwischenpause (\timePauseTwo)}
	\begin{itemize}
		\item Reihenfolge ÄA festlegen (weitreichendste zuerst) [\EP]
		\item \STADTVERTRETER\ in Empfang nehmen [\KOM]
		\item gegebenenfalls Fraktionsvorstand bei Rede unterstützen [Fraktionsteamer:innen]
	\end{itemize}
	
	\subsection{Plenarsitzung (\timePlenar)}
	\begin{enumerate}
		\item Grußwort \newline
        [\STADTVERTRETER]
		\item Plenardebatte [\EP]
		\begin{enumerate}
			\item Für jeden ÄA
			\begin{itemize}
				\item Vorstellung ÄA durch Fraktionsvize
				\item Reaktion durch übrige Fraktionsvize (Uhrzeigersinn)
				\item Abstimmung
				\item Zählen und Eintragen in Dashboard \newline (Mehr JA als NEIN Stimmen)
			\end{itemize}
			\item Alle angenommenen ÄA \& fertigen Gesetzentwurf vorstellen
			\item Abschlussrede durch Fraktionsvorsitz
			\item Abstimmung Gesetzesentwurf
			\item Zählen und Eintragen in Dashboard (Mehr JA als NEIN Stimmen)
		\end{enumerate}
		\item Vorstellung KV
		\item Danksagung an
		\begin{itemize}
            \item \STADTVERTRETER
			\item \EUVERTRETER
			\item \VERANSTALTUNGSORT
			\item \FOERDERER
		\end{itemize}
	\end{enumerate}
	
	\subsection{Nachbereitung}
	\begin{itemize}
		\item Aufräumen
		\item Gruppenfoto
		\item Feedbackrunde
		\item Bericht (Website) \& Insta-Post
		\item Dankesmail an \STADTVERTRETER
		\item Dankesmail an \EUVERTRETER
		\item Dankesmail an Lehrkräfte
	\end{itemize}
    \newpage
    \section{Packliste}
    \begin{itemize}
        \item Armbänder Fotoaufnahmen
        \item EU- \& JEF-Flaggen
        \item Klebeband
        \item Mappen
        \begin{itemize}
			\item Flyer
			\item Fraktionspapiere
			\item Gesetzesentwürfe
			\item Länderpapiere
        \end{itemize}
        \item Merch
        \item Namensschilder
        \item Raumschilder
        \item Schere
        \item Stifte        
        \item Teilnahmezertifikate
    \end{itemize}
\begin{comment}
    \newpage
    
    \section{Zeitstrahl}
    \begin{itemize}
        \item Im Jahr davor: Finanzierung
        \item Im Jahr davor: Ortfestlegung (Räumlichkeiten müssen stehen)
        \item Je nach Finanzierung: Terminfestlegung
        \item Direkt danach (mindestens halbes Jahr zuvor)
        \begin{itemize}
            \item Gastredner:innen anfragen
            \item Teamer:innen anfragen
            \item Schulklassen anfragen
            \item Hotel anfragen (mglw. Outsourcen)
        \end{itemize}
        \item 2-3 Monate davor: fixe Zusage der Teamer:innen einholen
        \item Direkt danach: Hotel buchen
        \item 1 Monat davor:
        \begin{itemize}
            \item Reminder an Lehrkräfte
            \item Catering abklären (mglw. Outsourcen)
            \item Aktualisierung der Unterlagen abgeschlossen
            \item mglw. Online-Schulung anbieten
        \end{itemize}
        \item 2 Wochen davor: 
        \begin{itemize}
            \item Restaurant reservieren (mglw. Outsourcen)
            \item KV Vorstellung abklären
        \end{itemize}
        \item In der Woche davor:
        \begin{itemize}
            \item Unterlagen vorbereiten
            \item Pressemitteilung rausschicken (mglw. Outsourcen)
        \end{itemize}
        \item Nach der SimEP: Nachbereitung
    \end{itemize}
\end{comment}
\end{document}