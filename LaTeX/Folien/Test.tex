%\documentclass{article}
%\usepackage{datatool}
%
%
%% Define the \party command
%\newcommand{\party}{PartyA}
%
%% Load the CSV data
%\DTLloaddb{seatsdata}{data_\party.csv}
%
%
%% Define a command to get the seat count
%\newcommand{\getseatcount}{
%    \DTLforeach*{seatsdata}{\partycol=party,\seatscol=seats}{%
%        \ifx\party\partycol%
%            \seatscol%
%            \DTLbreak%
%        \fi%
%    }
%}
%
%\begin{document}
%
%% Set the party
%\renewcommand{\party}{PartyA}
%
%% Get the seat count for the specified party
%The seat count for \party{} is \getseatcount.
%
%\end{document}

\documentclass{article}
\usepackage{datatool}
\usepackage{trimspaces}

% Load the CSV file
\DTLloaddb{caucus}{../../Daten/caucus_data.csv}
\newcommand{\group}{EPP}
% Define the custom command to print seats for a given party
\newcommand{\partyseats}[1]{%
    \DTLforeach{caucus}{\party=party,\total=total}{%
        \ifthenelse{\equal{#1}{\party}}{\total}{}%
    }%
}

\begin{document}

% Usage of the custom command
\group\ has \partyseats{\group} seats.

\end{document}
