\documentclass{beamer}
\usepackage{JEFtheme}
\usepackage{caption}
\usepackage{expl3} 
\usepackage{etoolbox}
\usepackage{tikz}
\usepackage{verbatim}
\usepackage{xstring}
\usepackage[german]{babel}
\usetikzlibrary{calc}
\newcommand\Fraktion{PfE}

\newcommand\Thema{Migration}
\newcommand\city{München}
\newcommand\datum{27.04.2025}
\newcommand\timeEinf{09:00-09:45}
\newcommand\timeFrakOne{09:45-11:15}
\newcommand\timeAuss{11:30-12:45}
\newcommand\timeMittag{12:45-13:15}
\newcommand\timeFrakTwo{13:15-13:45}
\newcommand\timePlenar{14:00-15:00}
\newcommand\politiker{Maria Noichl}
\newcommand\politikerOffice{Mitglied des Europäischen Parlaments}
\newcommand\stadtvertreter{Florian Kraus}
\newcommand\stadtvertreterOffice{Stadtschulrat}
\newcommand\localSupport{Europe Direct München}
\newcommand\sponsor{Stadt München}
\newcommand\jefvorsitz{Farras Fathi}
\newcommand\gendervorsitz{Landesvorsitzender}
\newcommand\evpLeader{TBD}
\newcommand\evpRoom{TBD}
\newcommand\sdLeader{TBD}
\newcommand\sdRoom{TBD}
\newcommand\reLeader{TBD}
\newcommand\reRoom{TBD}
\newcommand\greenLeader{TBD}
\newcommand\greenRoom{TBD}
\newcommand\pfeLeader{TBD}
\newcommand\pfeRoom{TBD}


%Information to be included in the title page:
\title{Simulation des Europäischen Parlaments}
\subtitle{Plenarsitzung}
\date{\datum}

\begin{document}
\frame{\titlepage}

\begin{frame}{Grußwort \stadtvertreter\ (\stadtvertreterOffice)}
\vspace{-1.5cm}
    \begin{center}
        \includegraphics[height=4cm]{Bilder/Stadt\_\city.png}
    \end{center}
\end{frame}

\begin{frame}{Agenda Plenarsitzung}
\vspace{-1.5cm}
    \begin{enumerate}
        \item Finale Änderungsanträge
        \begin{itemize}
            \item Fraktionsvize der entsprechenden Fraktion begründet diesen kurz (ca. 30 Sek.)
            \item Fraktionsvize der anderen Fraktionen nehmen im Uhrzeigersinn kurz Stellung (ca. 10 Sek.)
            \item Abstimmung über den Änderungsantrag
        \end{itemize}
        \item Abschlussreden der Fraktionsvorsitzenden (ca. 1 Min.)
        \item Abstimmung über gesamten Gesetzesentwurf
    \end{enumerate}
\end{frame}


\end{document}