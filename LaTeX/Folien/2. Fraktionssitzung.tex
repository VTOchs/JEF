\documentclass{beamer}
\usepackage{JEFtheme}
\usepackage{caption}
\usepackage{csvsimple}
\usepackage{expl3} 
\usepackage{etoolbox}
\usepackage{xstring}
\usepackage[german]{babel}
\newcommand\Fraktion{PfE}

\newcommand\Thema{Migration}
\newcommand\city{München}
\newcommand\datum{27.04.2025}
\newcommand\timeEinf{09:00-09:45}
\newcommand\timeFrakOne{09:45-11:15}
\newcommand\timeAuss{11:30-12:45}
\newcommand\timeMittag{12:45-13:15}
\newcommand\timeFrakTwo{13:15-13:45}
\newcommand\timePlenar{14:00-15:00}
\newcommand\politiker{Maria Noichl}
\newcommand\politikerOffice{Mitglied des Europäischen Parlaments}
\newcommand\stadtvertreter{Florian Kraus}
\newcommand\stadtvertreterOffice{Stadtschulrat}
\newcommand\localSupport{Europe Direct München}
\newcommand\sponsor{Stadt München}
\newcommand\jefvorsitz{Farras Fathi}
\newcommand\gendervorsitz{Landesvorsitzender}
\newcommand\evpLeader{TBD}
\newcommand\evpRoom{TBD}
\newcommand\sdLeader{TBD}
\newcommand\sdRoom{TBD}
\newcommand\reLeader{TBD}
\newcommand\reRoom{TBD}
\newcommand\greenLeader{TBD}
\newcommand\greenRoom{TBD}
\newcommand\pfeLeader{TBD}
\newcommand\pfeRoom{TBD}


% -------------------------------------------------------------

\IfEqCase{\Fraktion}{%
	{EVP}{
		\newcommand{\Fraktionsname}{Europäische Volkspartei}
	    \newcommand{\Fraktionskuerzel}{EVP}
	}%
	{SD}{
		\newcommand{\Fraktionsname}{Progressive Allianz der Sozialdemokraten}
        \newcommand{\Fraktionskuerzel}{S\&D}
	}%
	{RE}{
		\newcommand{\Fraktionsname}{Renew Europe}
		\newcommand{\Fraktionskuerzel}{Renew}
	}%
	{Green}{
		\newcommand{\Fraktionsname}{Die Grünen/Europäische Freie Allianz}
		\newcommand{\Fraktionskuerzel}{Grüne/EFA}
	}%
	{PfE}{
		\newcommand{\Fraktionsname}{Patrioten für Europa}
		\newcommand{\Fraktionskuerzel}{PfE}
	}%
}


%Information to be included in the title page:
\title{\Fraktionsname\ (\Fraktionskuerzel)}
\subtitle{2. Fraktionssitzung}
\date{\datum}

\begin{document}

\frame{\titlepage}

\begin{frame}{Ausschusssitzungen} % Besprechung Ausschusssitzungen
\vspace{-1.5cm}
Berichtet aus jedem Ausschuss:
\newline
\begin{itemize}
    \item Habt ihr euren Änderungsantrag durchbekommen?
    \item Welche anderen Fraktionen haben ihre Änderungsanträge durchbekommen?
    \item Wer hat mit wem zusammengearbeitet?
    \item Ist der Gesetzesentwurf nun mehr in unserem Sinne oder weniger?
\end{itemize}
\end{frame}

\begin{frame}{Finaler Änderungsantrag} % Festlegung finaler ÄA
\vspace{-0.5cm}
In der Plenarsitzung können wir als Fraktion noch einen (!) finalen Änderungsantrag stellen. Beispielsweise: \newline
\begin{itemize}
    \item Alter Änderungsantrag, der im Ausschuss abgelehnt wurde
    \item Verworfene Idee aus erster Fraktionssitzung
    \item Rückgängigmachen von Änderungsantrag, der uns überhaupt nicht passt
\end{itemize}
\end{frame}

\begin{frame}{Rede Fraktionsvorsitzende}
\vspace{-1.5cm}
Falls die beiden Fraktionsvorsitzenden noch nicht wissen, was sie sagen sollen, helfen wir euch gerne!
\end{frame}


\end{document}